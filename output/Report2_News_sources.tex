% Options for packages loaded elsewhere
\PassOptionsToPackage{unicode}{hyperref}
\PassOptionsToPackage{hyphens}{url}
%
\documentclass[
]{article}
\title{The relationship between trust in institutions and use of news
media sources in the early phase of the COVID-19 pandemic}
\usepackage{etoolbox}
\makeatletter
\providecommand{\subtitle}[1]{% add subtitle to \maketitle
  \apptocmd{\@title}{\par {\large #1 \par}}{}{}
}
\makeatother
\subtitle{An analysis based on the GESIS Panel Special Survey on the
Coronavirus SARS-CoV-2 Outbreak in Germany}
\author{R User}
\date{2021-11-18}

\usepackage{amsmath,amssymb}
\usepackage{lmodern}
\usepackage{iftex}
\ifPDFTeX
  \usepackage[T1]{fontenc}
  \usepackage[utf8]{inputenc}
  \usepackage{textcomp} % provide euro and other symbols
\else % if luatex or xetex
  \usepackage{unicode-math}
  \defaultfontfeatures{Scale=MatchLowercase}
  \defaultfontfeatures[\rmfamily]{Ligatures=TeX,Scale=1}
\fi
% Use upquote if available, for straight quotes in verbatim environments
\IfFileExists{upquote.sty}{\usepackage{upquote}}{}
\IfFileExists{microtype.sty}{% use microtype if available
  \usepackage[]{microtype}
  \UseMicrotypeSet[protrusion]{basicmath} % disable protrusion for tt fonts
}{}
\makeatletter
\@ifundefined{KOMAClassName}{% if non-KOMA class
  \IfFileExists{parskip.sty}{%
    \usepackage{parskip}
  }{% else
    \setlength{\parindent}{0pt}
    \setlength{\parskip}{6pt plus 2pt minus 1pt}}
}{% if KOMA class
  \KOMAoptions{parskip=half}}
\makeatother
\usepackage{xcolor}
\IfFileExists{xurl.sty}{\usepackage{xurl}}{} % add URL line breaks if available
\IfFileExists{bookmark.sty}{\usepackage{bookmark}}{\usepackage{hyperref}}
\hypersetup{
  pdftitle={The relationship between trust in institutions and use of news media sources in the early phase of the COVID-19 pandemic},
  pdfauthor={R User},
  hidelinks,
  pdfcreator={LaTeX via pandoc}}
\urlstyle{same} % disable monospaced font for URLs
\usepackage[margin=1in]{geometry}
\usepackage{longtable,booktabs,array}
\usepackage{calc} % for calculating minipage widths
% Correct order of tables after \paragraph or \subparagraph
\usepackage{etoolbox}
\makeatletter
\patchcmd\longtable{\par}{\if@noskipsec\mbox{}\fi\par}{}{}
\makeatother
% Allow footnotes in longtable head/foot
\IfFileExists{footnotehyper.sty}{\usepackage{footnotehyper}}{\usepackage{footnote}}
\makesavenoteenv{longtable}
\usepackage{graphicx}
\makeatletter
\def\maxwidth{\ifdim\Gin@nat@width>\linewidth\linewidth\else\Gin@nat@width\fi}
\def\maxheight{\ifdim\Gin@nat@height>\textheight\textheight\else\Gin@nat@height\fi}
\makeatother
% Scale images if necessary, so that they will not overflow the page
% margins by default, and it is still possible to overwrite the defaults
% using explicit options in \includegraphics[width, height, ...]{}
\setkeys{Gin}{width=\maxwidth,height=\maxheight,keepaspectratio}
% Set default figure placement to htbp
\makeatletter
\def\fps@figure{htbp}
\makeatother
\setlength{\emergencystretch}{3em} % prevent overfull lines
\providecommand{\tightlist}{%
  \setlength{\itemsep}{0pt}\setlength{\parskip}{0pt}}
\setcounter{secnumdepth}{-\maxdimen} % remove section numbering
\newlength{\cslhangindent}
\setlength{\cslhangindent}{1.5em}
\newlength{\csllabelwidth}
\setlength{\csllabelwidth}{3em}
\newlength{\cslentryspacingunit} % times entry-spacing
\setlength{\cslentryspacingunit}{\parskip}
\newenvironment{CSLReferences}[2] % #1 hanging-ident, #2 entry spacing
 {% don't indent paragraphs
  \setlength{\parindent}{0pt}
  % turn on hanging indent if param 1 is 1
  \ifodd #1
  \let\oldpar\par
  \def\par{\hangindent=\cslhangindent\oldpar}
  \fi
  % set entry spacing
  \setlength{\parskip}{#2\cslentryspacingunit}
 }%
 {}
\usepackage{calc}
\newcommand{\CSLBlock}[1]{#1\hfill\break}
\newcommand{\CSLLeftMargin}[1]{\parbox[t]{\csllabelwidth}{#1}}
\newcommand{\CSLRightInline}[1]{\parbox[t]{\linewidth - \csllabelwidth}{#1}\break}
\newcommand{\CSLIndent}[1]{\hspace{\cslhangindent}#1}
\usepackage{flafter}
\ifLuaTeX
  \usepackage{selnolig}  % disable illegal ligatures
\fi

\begin{document}
\maketitle

\hypertarget{research-question}{%
\section{Research question}\label{research-question}}

We want to investigate whether the sources of information that people
have used for getting current information about the Corona virus are
related to their trust in different groups and institutions.

\hypertarget{methods}{%
\section{Methods}\label{methods}}

We used R version 4.1.2 (2021-11-01) (R Core Team, 2021) for all data
wrangling and analyses.

\hypertarget{sample}{%
\subsection{Sample}\label{sample}}

The data we use comes from the \emph{Public Use File (PUF) of the GESIS
Panel Special Survey on the Coronavirus SARS-CoV-2 Outbreak in Germany}
(GESIS Panel Team, 2020). The data set includes data for \emph{N} = 3765
respondents, of which \emph{n} = 1832 were female and \emph{n} = 1933
were male.

\hypertarget{measures}{%
\subsection{Measures}\label{measures}}

The variables we are interested in for this analysis are the ones
measuring trust in 1) the federal government, 2) the World Health
Organization (WHO), and 3) scientists in dealing with the Corona virus
as predictors, and the use of a) national public television or radio and
b) Facebook as sources of current information about the Corona virus.

The predictors assessing trust were measured on a 5-point scale ranging
from ``1 - Do not trust at all'' to ``5 - Entirely trust,'' with the
additional response option ``Don't know'' (those responses were excluded
from our analyses). The two outcome variables were binary items.
Respondents could either indicate that they used the respective source
(in our case: national public television or radio and Facebook) for
getting current information about the Corona virus or not. All data
wrangling was done with the \texttt{dplyr} package (Wickham et al.,
2021) which is part of the \texttt{tidyverse} (Wickham et al., 2019).

\newpage

\hypertarget{results}{%
\section{Results}\label{results}}

\hypertarget{descriptive-statistics}{%
\subsection{Descriptive statistics}\label{descriptive-statistics}}

\emph{Table 1} shows the descriptive statistics for the trust variables.

\begin{table}[!htbp] \centering 
  \caption{Descriptive statistics} 
  \label{Descriptive statistics} 
\begin{tabular}{@{\extracolsep{5pt}}lccccccc} 
\\[-1.8ex]\hline 
\hline \\[-1.8ex] 
Statistic & \multicolumn{1}{c}{N} & \multicolumn{1}{c}{Mean} & \multicolumn{1}{c}{St. Dev.} & \multicolumn{1}{c}{Min} & \multicolumn{1}{c}{Pctl(25)} & \multicolumn{1}{c}{Pctl(75)} & \multicolumn{1}{c}{Max} \\ 
\hline \\[-1.8ex] 
Trust in federal government & 2,761 & 3.66 & 0.99 & 1.00 & 3.00 & 4.00 & 5.00 \\ 
Trust in WHO & 2,731 & 4.03 & 0.91 & 1.00 & 4.00 & 5.00 & 5.00 \\ 
Trust in scientists & 2,728 & 4.26 & 0.76 & 1.00 & 4.00 & 5.00 & 5.00 \\ 
\hline \\[-1.8ex] 
\end{tabular} 
\end{table}

\emph{Note}: The table was produced with the \texttt{stargazer} package
(Hlavac, 2018).

\hypertarget{correlations}{%
\subsection{Correlations}\label{correlations}}

\emph{Table 2} shows the correlations between the perceived personal
risk variables. Correlations were computed using the \texttt{corr}
package (Kuhn et al., 2020).

\begin{longtable}[]{@{}llll@{}}
\caption{Correlations between the trust variables}\tabularnewline
\toprule
Measure & 1 & 2 & 3 \\
\midrule
\endfirsthead
\toprule
Measure & 1 & 2 & 3 \\
\midrule
\endhead
Trust in federal government & --- & --- & --- \\
Trust in WHO & .47 & --- & --- \\
Trust in scientists & .35 & .44 & --- \\
\bottomrule
\end{longtable}

\newpage

\hypertarget{regression-analysis}{%
\subsection{Regression analysis}\label{regression-analysis}}

We use a logistic regression model to explore how the use of 1) national
public television and radio and 2) Facebook as sources of current
information about the Corona virus are predicted by trust in the federal
government, the WHO, and scientists in general.

The general formula for a logistic regression model with three
predictors is
\(Pr(Y_i=1|X_i) = {\frac{exp(\beta_0 + \beta_1X_i + \beta_2X_2 + \beta_3X_3}{1 + exp (\beta_0 + \beta_1X_i + \beta_2X_2 + \beta_3X_3}}\)

Accordingly, in our case, the formulas for the two models are:

\begin{equation}
\log\left[ \frac { P( \operatorname{info\_nat\_pub\_br} = \operatorname{1} ) }{ 1 - P( \operatorname{info\_nat\_pub\_br} = \operatorname{1} ) } \right] = \beta_{0} + \beta_{1}(\operatorname{trust\_government}) + \beta_{2}(\operatorname{trust\_who}) + \beta_{3}(\operatorname{trust\_scientists})
\end{equation}

\begin{equation}
\log\left[ \frac { P( \operatorname{info\_fb} = \operatorname{0} ) }{ 1 - P( \operatorname{info\_fb} = \operatorname{0} ) } \right] = \beta_{0} + \beta_{1}(\operatorname{trust\_government}) + \beta_{2}(\operatorname{trust\_who}) + \beta_{3}(\operatorname{trust\_scientists})
\end{equation}

\emph{Note}: To create the formulas for our models, we used the
\texttt{equatiomatic} package (Anderson et al., 2021).

\emph{Table 3} shows the results of the two logistic regression models.

\begin{table}[!htbp] \centering 
  \caption{Results of the two logistic regression models} 
  \label{Regression results} 
\begin{tabular}{@{\extracolsep{5pt}}lcc} 
\\[-1.8ex]\hline 
\hline \\[-1.8ex] 
 & \multicolumn{2}{c}{\textit{Dependent variable:}} \\ 
\cline{2-3} 
\\[-1.8ex] & Public broadcasting as info source & Facebook as info source \\ 
\\[-1.8ex] & (1) & (2)\\ 
\hline \\[-1.8ex] 
 Trust in federal government & 0.528$^{***}$ & $-$0.193$^{***}$ \\ 
  & (0.066) & (0.052) \\ 
  & & \\ 
 Trust in WHO & $-$0.014 & 0.066 \\ 
  & (0.075) & (0.061) \\ 
  & & \\ 
 Trust in scientists & 0.167$^{**}$ & $-$0.133$^{**}$ \\ 
  & (0.080) & (0.065) \\ 
  & & \\ 
 Constant & $-$0.224 & $-$0.467$^{*}$ \\ 
  & (0.325) & (0.269) \\ 
  & & \\ 
\hline \\[-1.8ex] 
Observations & 3,027 & 3,027 \\ 
Log Likelihood & $-$927.508 & $-$1,459.690 \\ 
Akaike Inf. Crit. & 1,863.016 & 2,927.379 \\ 
\hline 
\hline \\[-1.8ex] 
\textit{Note:}  & \multicolumn{2}{r}{$^{*}$p$<$0.1; $^{**}$p$<$0.05; $^{***}$p$<$0.01} \\ 
\end{tabular} 
\end{table}

\emph{Note}: The table was produced with the \texttt{stargazer} package
(Hlavac, 2018).

\hypertarget{discussion}{%
\section{Discussion}\label{discussion}}

Trust in the federal government and trust in scientists emerged as
significant positive predictors of using national public broadcasting as
a source of current information about the Corona virus, meaning that
people who show higher trust in these institutions are more likely to
use this source of information. Contrarily, these two trust measures
negatively predicted the use of Facebook as a source of information,
meaning that individuals with lower levels of trust in the federal
government and scientists are more likely to use Facebook as an
information source. These results show that trust in institutions and
information behavior are related.

\hypertarget{references}{%
\section{References}\label{references}}

\hypertarget{refs}{}
\begin{CSLReferences}{1}{0}
\leavevmode\vadjust pre{\hypertarget{ref-R-equatiomatic}{}}%
Anderson, D., Heiss, A., \& Sumners, J. (2021). \emph{Equatiomatic:
Transform models into LaTeX equations}.
\url{https://CRAN.R-project.org/package=equatiomatic}

\leavevmode\vadjust pre{\hypertarget{ref-GESISPanelTeam2020}{}}%
GESIS Panel Team. (2020). \emph{{GESIS Panel Special Survey on the
Coronavirus SARS-CoV-2 Outbreak in GermanyGESIS Panel Special Survey on
the Coronavirus SARS-CoV-2 Outbreak in Germany}}. {GESIS Data Archive}.
\url{https://doi.org/10.4232/1.13520}

\leavevmode\vadjust pre{\hypertarget{ref-R-stargazer}{}}%
Hlavac, M. (2018). \emph{Stargazer: Well-formatted regression and
summary statistics tables}.
\url{https://CRAN.R-project.org/package=stargazer}

\leavevmode\vadjust pre{\hypertarget{ref-R-corrr}{}}%
Kuhn, M., Jackson, S., \& Cimentada, J. (2020). \emph{Corrr:
Correlations in r}. \url{https://CRAN.R-project.org/package=corrr}

\leavevmode\vadjust pre{\hypertarget{ref-R}{}}%
R Core Team. (2021). \emph{R: A language and environment for statistical
computing}. R Foundation for Statistical Computing.
\url{https://www.R-project.org/}

\leavevmode\vadjust pre{\hypertarget{ref-tidyverse2019}{}}%
Wickham, H., Averick, M., Bryan, J., Chang, W., McGowan, L. D.,
François, R., Grolemund, G., Hayes, A., Henry, L., Hester, J., Kuhn, M.,
Pedersen, T. L., Miller, E., Bache, S. M., Müller, K., Ooms, J.,
Robinson, D., Seidel, D. P., Spinu, V., \ldots{} Yutani, H. (2019).
Welcome to the {tidyverse}. \emph{Journal of Open Source Software},
\emph{4}(43), 1686. \url{https://doi.org/10.21105/joss.01686}

\leavevmode\vadjust pre{\hypertarget{ref-R-dplyr}{}}%
Wickham, H., François, R., Henry, L., \& Müller, K. (2021). \emph{Dplyr:
A grammar of data manipulation}.
\url{https://CRAN.R-project.org/package=dplyr}

\end{CSLReferences}

\hypertarget{reproducibility-information}{%
\section{Reproducibility
information}\label{reproducibility-information}}

\begin{itemize}\raggedright
  \item R version 4.1.2 (2021-11-01), \verb|x86_64-w64-mingw32|
  \item Running under: \verb|Windows 10 x64 (build 18362)|
  \item Matrix products: default
  \item Base packages: base, datasets, graphics, grDevices, methods, stats, utils
  \item Other packages: corrr~0.4.3, dplyr~1.0.6, equatiomatic~0.3.0, forcats~0.5.1,
    fs~1.5.0, ggplot2~3.3.3, purrr~0.3.4, readr~1.4.0, stargazer~5.2.2, stringr~1.4.0,
    tibble~3.1.2, tidyr~1.1.3, tidyverse~1.3.1
  \item Loaded via a namespace (and not attached): assertthat~0.2.1, backports~1.2.1,
    broom~0.7.6, bslib~0.2.5.1, cellranger~1.1.0, cli~2.5.0, colorspace~2.0-1,
    compiler~4.1.2, crayon~1.4.1, DBI~1.1.1, dbplyr~2.1.1, digest~0.6.27, ellipsis~0.3.2,
    evaluate~0.14, fansi~0.4.2, fastmap~1.1.0, generics~0.1.0, glue~1.4.2, grid~4.1.2,
    gtable~0.3.0, haven~2.4.3, highr~0.9, hms~1.1.0, htmltools~0.5.1.1, httpuv~1.6.1,
    httr~1.4.2, jquerylib~0.1.4, jsonlite~1.7.2, knitr~1.33, later~1.2.0, lifecycle~1.0.0,
    lubridate~1.7.10, magrittr~2.0.1, mime~0.10, modelr~0.1.8, munsell~0.5.0, pillar~1.6.1,
    pkgconfig~2.0.3, promises~1.2.0.1, R6~2.5.0, Rcpp~1.0.6, readxl~1.3.1, reprex~2.0.0,
    rlang~0.4.11, rmarkdown~2.11, rstudioapi~0.13, rvest~1.0.0, sass~0.4.0, scales~1.1.1,
    shiny~1.6.0, stringi~1.6.2, tidyselect~1.1.1, tinytex~0.31, tools~4.1.2, utf8~1.2.1,
    vctrs~0.3.8, withr~2.4.2, xfun~0.23, xml2~1.3.2, xtable~1.8-4, yaml~2.2.1
\end{itemize}

\end{document}
